\chapter*{ВВЕДЕНИЕ}							% Заголовок
\addcontentsline{toc}{chapter}{ВВЕДЕНИЕ}	% Добавляем его в оглавление


\newcommand{\actuality}{\textbf{Актуальность исследования.}}
\newcommand{\aim}{\textbf{Целью}}
\newcommand{\tasks}{\textbf{задачи}}
\newcommand{\defpositions}{\textbf{Основные положения, выносимые на~защиту:}}
\newcommand{\novelty}{\textbf{Научная новизна:}}
\newcommand{\influence}{\textbf{Научная и практическая значимость}}
\newcommand{\reliability}{\textbf{Степень достоверности}}
\newcommand{\probation}{\textbf{Апробация работы.}}
\newcommand{\contribution}{\textbf{Личный вклад.}}
\newcommand{\publications}{\textbf{Публикации}}

{\actuality} Мировое энергетическое состояние такого, что человечество потребляет всё больше и больше энергии. Однако принципиально новые источники энергии до сих пор не могут выйти на передний план и это происходит по многим причинам: от экономических до инженерных и теоретических проблем. 

Одним из таких источников является управляемый термоядерный синтез (УТС) --- получение энергии при очень сильном разогреве газа из легких элементов и их последующим синтезом. 

Температура, при которой происходит такая реакция лежит в пределах от 10 кэВ до 100 кэВ. Основная проблема удержания происходит как раз из-за таких огромных температур, ведь чуть ли не единственный способ удержания в данном случае ---  сильные неоднородные электромагнитные поля. Данная работа посвящена в большей степени магнитным ловушка открытого типа.



\aim\ данной работы является исследование применимости метода молекулярной динамики к задачам моделирования субтермоядерной плазмы в магнитных ловушках открытого типа.

Для~достижения поставленной цели необходимо было решить следующие {\tasks}:
\begin{enumerate}
  \item Изучить современное состояние установок по удержанию горячей плазмы, а также состояние численных методов, применяемых при решении данного класса задач задач.
  \item Составить математическую модель на основе метода крупных частиц, которая будет учитывать необходимые физические процессы.
  \item Численная реализация составленной математической модели.
  \item Проведение серий численных экспериментов по моделированию горячей плазмы в открытой магнитной ловушке.
  \item Обработка и анализ полученных в ходе численных экспериментов данных и их сопоставление с результатами, полученными на экспериментальной установке другими авторами.
\end{enumerate}
%old:
%\begin{enumerate}
%  \item Изучить современное состояние установок по удержанию горячей плазмы.
%  \item Изучить современное состояние численных методов, применяемых при решении подобных задач.
%  \item Составить математическую модель на основе метода молекулярной динамики, которая будет учитывать необходимые физические процессы.
%  \item Численная реализация составленной математической модели.
%  \item Проверка адекватности составленной численной модели. Анализ данных. Сравнение полученных данных с экспериментом.
%\end{enumerate}

\defpositions
\begin{enumerate}
  \item Разработка модели и вычислительной программы для моделирования горячей релятивистской плазмы методом молекулярной динамики.
  \item Результаты численных экспериментов по моделированию горячей плазмы в открытой магнитной ловушке.
\end{enumerate}

% Научной новизны быть не должно. Да её и нет особо.
\novelty \ 
было выполнено оригинальное исследование о применимости метода молекулярной динамики в рамках моделирования крупногабаритных ловушек открытого типа.


\influence:\ установлен факт возможности применения метода молекулярной динамики для оценки функции распределения плотности в пространстве открытой магнитной ловушки.

\reliability\ полученных результатов обеспечивается использованием подходов из первых принципов. Результаты находятся в соответствии с результатами, полученными другими авторами --- как теоретиками, так и экспериментаторами.

\probation\
Основные результаты работы докладывались автором на всероссийской научной конференции студентов-физиков и молодых учёных ВНКСФ-22 (2016 г.) и на смотр--конкурсе научных, конструкторских и технологических работ студентов ВолгГТУ (2016 г.).


\publications\textbf{:}
\begin{enumerate}
	\item Тофтул, И. Д. Моделирование горячей плазмы в ГДЛ методом молекулярной динамики [Текст] / И. Д. Тофтул, Д. Г. Ковтун // Двадцать вторая Всероссийская научная конференция студентов-физиков и молодых учёных. ВНКСФ-22 (Ростов--на--Дону, 21 -- 28 апреля 2016 г.): тез. докл. / Ин-т электрофизики УрО РАН, Ассоциация студентов-физиков и молодых учёных России. – Ростов-на-Дону, 2016. --- С. 215--216.
\end{enumerate}
 % Характеристика работы по структуре во введении и в автореферате не отличается (ГОСТ Р 7.0.11, пункты 5.3.1 и 9.2.1), потому её загружаем из одного и того же внешнего файла, предварительно задав форму выделения некоторым параметрам

\textbf{Объем и структура работы.} Бакалавровская диссертация состоит из~введения, четырёх глав и заключения.
%~\textbf{??} приложений.
%% на случай ошибок оставляю исходный кусок на месте, закомментированным
%Полный объём диссертации составляет  \ref*{TotPages}~страницу с~\totalfigures{}~рисунками и~\totaltables{}~таблицами. Список литературы содержит \total{citenum}~наименований.
%
Полный объём выпускной работы составляет \formbytotal{TotPages}{страниц}{у}{ы}{} 
с~\formbytotal{totalcount@figure}{рисунк}{ом}{ами}{ами}. 
Список литературы содержит  
\formbytotal{citenum}{наименован}{ие}{ия}{ий}.
