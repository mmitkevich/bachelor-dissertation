\chapter*{РЕФЕРАТ}             % Заголовок

Произведен литературный обзор установок различного типа по удержанию горячей плазмы, а также численных методов, применяемых при моделирования рассмотренных систем. Написан ряд программ для моделирования одиночной релятивистской заряженной частицы в электромагнитных полях, а также ансамбля релятивистских частиц. Был выбран метод <<частица-частица>> Поле пространственного заряда рассчитывается как с учетом времен запаздывания, так и без. Написана программа для расчета магнитного поля от системы распределенных токов при аксиальной симметрии. Написан ряд программ, обеспечивающих обработку численных данных.

Проведено сравнение численных решений с аналитическим решениями в случае одиночного электрона, с теоретическими работами других авторов при моделировании потоков электронов в скрещенных полях, а также с экспериментальными данными других авторов, полученными на установке ГДЛ. Было получено соответствие результатов.

\textbf{Ключевые слова:} релятивистская плазма, релятивистский электронных поток, метод <<частица-частица>>, метод крупных частиц, PIC, сеточные методы, LBM, ГДЛ, открытая магнитная ловушка, токамак, пробкотрон, MPI, OpenMP.