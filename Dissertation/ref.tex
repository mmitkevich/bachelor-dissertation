\chapter*{РЕФЕРАТ}             

Произведен обзор установок различного типа по удержанию горячей плазмы, а также численных методов, применяемых при моделирования рассмотренных систем. Написан ряд программ для моделирования одиночной релятивистской заряженной частицы в электромагнитных полях, а также ансамбля релятивистских частиц (методом PP). Поле пространственного заряда рассчитывается как с учетом времен запаздывания, так и без. Написана программа для расчета магнитного поля от системы распределенных токов при аксиальной симметрии. Написан ряд программ, обеспечивающих обработку численных данных.

Полученная программа позволяет определить распределение плотности и скоростей частиц в различных открытых магнитных ловушках. Проведено сравнение численных решений с аналитическим решениями (одиночный электрон) и работами других авторов (электронный поток), а также с экспериментальными данными других авторов, полученными на установке ГДЛ. Было получено соответствие результатов.

\textbf{Ключевые слова:} релятивистская плазма, релятивистский электронных поток, метод <<частица-частица>>, метод крупных частиц, PIC, сеточные методы, LBM, ГДЛ, открытая магнитная ловушка, токамак, пробкотрон, MPI, OpenMP.

\ 

Review about different devices designed to confine hot plasma with magnetic fields was made. Moreover, numerical methods for hot plasma modulation were also reviewed. A number of computer codes were written to modulate single charged particle, hot plasma (PP method had been chosen) and to handle and analyze all the output data. Besides, another code was written to calculate magnetic field in open trap.

Results were processed and analyzed. Positive comparison with a number of benchmakrs were made. Furthermore, a real experimental scenario of the GDL device was computed, qualitative agreement with experimental results was obtained.
Created model can be used to describe similar systems.


\textbf{Keywords:} relativistic plasma, relativistic electron flux, <<particle-particle>> method, enlargement method, PIC, Lattice Bolzmann Method, LBM, GDL, open magnetic trap, tokamak, magnetic mirror , MPI, OpenMP.