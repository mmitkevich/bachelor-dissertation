\chapter*{ЗАКЛЮЧЕНИЕ}						% Заголовок
\addcontentsline{toc}{chapter}{ЗАКЛЮЧЕНИЕ}	% Добавляем его в оглавление

%% Согласно ГОСТ Р 7.0.11-2011:
%% 5.3.3 В заключении диссертации излагают итоги выполненного исследования, рекомендации, перспективы дальнейшей разработки темы.
%% 9.2.3 В заключении автореферата диссертации излагают итоги данного исследования, рекомендации и перспективы дальнейшей разработки темы.
%% Поэтому имеет смысл сделать эту часть общей и загрузить из одного файла в автореферат и в диссертацию:

В процессе выполнения работы был произведён обзор современных установок по удержанию субтермоядерной плазмы, а также обзор численных методов, применяемых для описания подобных систем. Было выяснено, что есть возможность применения метода молекулярной динамики при решении задач больш\'{о}го масштаба.

Основные результаты работы заключаются в следующем:
%% Согласно ГОСТ Р 7.0.11-2011:
%% 5.3.3 В заключении диссертации излагают итоги выполненного исследования, рекомендации, перспективы дальнейшей разработки темы.
%% 9.2.3 В заключении автореферата диссертации излагают итоги данного исследования, рекомендации и перспективы дальнейшей разработки темы.
\begin{enumerate}
  \item Для выполнения поставленных задач был создан ряд многоцелевых вычислительных программ, написанных на языках \texttt{c++} и Python.
  \item Получена конфигурация магнитного поля открытой ловушки аналогичной используемой в ГДЛ.
  \item В результате проведения серии численных экспериментов по моделированию поведения горячей плазмы в открытой магнитной ловушке получен набор параметров (плотность, температура, распределение по скоростям), характеризующих динамику горячих ионов.
  \item Сопоставление результатов численных экспериментов с экспериментальными данными позволяет сделать вывод о применимости составленной модели в задачах, где требуется оценочные результаты. Модель позволяет качественно правильно рассчитать распределение концентрации в объёме установки, а также распределение частиц по скоростям.
\end{enumerate}


Развив данную модель, а именно добавив учёт процессов ионизации, термоядерных реакций, более точный расчёт внешнего электромагнитного поля (в том числе учесть дополнительные системы стабилизации), станет возможным исследовать более тонкие эффекты. Результаты данной работы могут служить отправной точкой в следующем этапе исследований.
