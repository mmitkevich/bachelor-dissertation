\chapter{ЧИСЛЕННОЕ РЕШЕНИЕ ОСНОВОПОЛАГАЮЩИХ УРАВНЕНИЙ} \label{ch3}


\section{Выбор математического ядра}


Возможны два основных подхода к численному решению релятивистского  уравнения движения. В качестве переменных можно выбрать либо координаты и скорости, либо координаты и импульсы частиц.
В приложении \ref{AppendixA} эти подходы рассмотрены подробно, здесь же приведены только рассуждения в пользу выбора того или иного решения.

С математической точки зрения решения \eqref{eq:style1} и \eqref{eq:style2} эквивалентны относительно конечного результата. Однако задача стоит в расстановке приоритетов с точки зрения их численного решения. 

Решение \eqref{eq:style1} или т.н. <<импульсное>> решение имеет меньшее количество математических операций для расчёта. С другой стороны в нём никак явным образом не отражено ограничение на скорость частицы.

Решение \eqref{eq:style2} же более трудоёмкое с точки зрения расчётов, однако в нём явным образом исключён факт превышения скорости $c$.

Численный эксперимент показал, что при моделировании электрона в сильном однородном электрическом поле и высокой начальной скорости <<импульсное>> в конце-концов расходится при $v \lesssim c$, а именно наступает момент, когда $v>c$, что физически неверно. Решение \eqref{eq:style2} при тех же условиях ведёт себя стабильно. Однако при низких релятивистских скоростях при одном и том же шаге временной дискретизации <<импульсный>> подход даёт более высокую точность решения.




\section{Численное решение уравнений магнитостатики}

В случае решения задач с помощью метода <<частица-частица>>, представляется затруднительным решать самосогласованные полевые задачи. Однако, режимов работы установок по удержанию плазмы, связанных с самосогласованностью электромагнитных полей не так уж и много. К ним относятся, например, эксперименты с компактным тором. В англоязычной литературе известно название Field Reversed Configuration (FRC) --- конфигурация с обращенным полем. Теория пока мало изучена, а постановка экспериментов крайне сложна \cite{Steinhauer2006, Ryzhkov2015d, Intrator2004}. Стоить отметить, что эксперименты по FRC проводятся именно на базах открытых ловушек.

\subsection{Общие положения}

Во многих задачах, особенно когда требуется получить оценочный результат, достаточно использовать уравнения магнитостатики. В случае $\partial \vec{E}/\partial t = 0$ пара уравнений максвелла становится закрытой системой:
\begin{numcases}{}
	\Div \vec{B} = 0, \\
	\Rot \vec{B} = \mu_0 \vec{j}. \label{eq:max_without_E}
\end{numcases}

А так как поле $\vec{B}$ можно представить через векторный потенциал как  $\vec{B} = \Rot \vec{A}$, то тогда, принимая во внимание формулу векторного анализа для двойного ротора, перепишем выражение \eqref{eq:max_without_E} в следующем виде
\begin{equation}
	\Grad \Div \vec{A} - \Delta \vec{A} = \mu_0 \vec{j}.
\end{equation}
Используя кулоновскую калибровку для векторного потенциала $\Div \vec{A} = 0$, получим
\begin{equation}
	\Delta \vec{A} = - \mu_0 \vec{j}. \label{eq:toSolveToFindB}
\end{equation}
Решение уравнения \eqref{eq:toSolveToFindB} можно записать в виде \cite{Landau2}
\begin{equation}
	\vec{A} = \frac{\mu_0}{4 \pi} \int \frac{\vec{j}}{R} dV,
\end{equation}
где $R$ --- радиус вектор, направленный их $dV$ в точку наблюдения. Найдя $\Rot \vec{A}$, получим так называемый закон Био--Савара
\begin{equation}
	\vec{B} = \frac{\mu_0}{4 \pi} \int \frac{\vecmult{j}{R}}{R^3} dV.
	\label{eq:bio_savar}
\end{equation}

\subsection{Случай тонкого кольца с током}

Учитывая относительные размеры и расположения магнитных катушек и камеры, где непосредственно находится плазма, можно сделать предположение, что магнитная катушка представляет собой тонкий контур с током в форме окружности радиуса $R'$. В этом случае задача значительно упрощается. 

Пусть имеется контур с током $\gamma = 2 \pi R'$, по которому течёт ток $I$. Учитывая связь $\vec{j} dV = I \vec{dl}$, перепишем выражение \eqref{eq:bio_savar} для витка с током
\begin{equation}
	\vec{B} = \frac{\mu_0}{4 \pi} \oint \limits_{\gamma = 2 \pi R'} \frac{I \cdot \vecmult{dl}{R}}{R^3}.
	\label{eq:bio_savar_curl}
\end{equation}

Перейдём в цилиндрическую систему координат $(r,\varphi,z)$. Ось $z$ направим перпендикулярно плоскости контура $\gamma$. Ток $I$, очевидно, можно вынести из под знака интеграла. Найдём модуль магнитной индукции, принимая во внимание тот факт, что у элемента $\vec{dl}$ есть только $\varphi$-я компонента:
\begin{equation}
	B = \frac{\mu_0 i}{4 \pi}  \int \limits_0^{2 \pi} \frac{R' d \varphi}{R^2} = \frac{\mu_0 R'}{2} \cdot \frac{I}{R^2}.
	\label{eq:calcBcyl}
\end{equation}
Направление вектора $\vec{B}$ будет определятся согласно правилу правого винта.
